\chapter{Marco Referencial}
\section{Introducci\'on}
A trav\'ez de la historia, el conocimiento humano ha sido almacenado en gran medida
de forma escrita, teniendose a los libros, peri\'odicos, art\'iculos cient\'ificos,
etc., como muestra de ello. En general este conocimiento ha sido escrito en lenguaje
natural. \\

Dada la abrumadora cantidad de este tipo de informaci\'on, es necesario su organizaci\'on
para identificarla entre las diferentes \'areas de conocimiento a las que puede 
pertenecer, para este cometido es necesario contar con criterios de clasificaci\'on 
que faciliten su posterior consulta. \\

El Procesamiento del Lenguaje Natural (PLN), es una rama de la Ling\"uistica 
Computacional que b\'asicamente estudia la comunicaci\'on entre personas y 
m\'aquinas por medio de lenguajes naturales. Entre las principales tareas de las que 
se ocupa, se encuentra la \emph{extracci\'on de informaci\'on} que es utilizada 
para el reconocimiento de nombres de entidades y extracci\'on de terminolog\'ia. \\

La extracci\'on de terminolog\'ia permite identificar y extraer \emph{palabras claves} 
de un texto analizado, que representan su tema o motivo central, las cuales representan
un criterio de clasificaci\'on eficiente.\\

Existen dos enfoques claramente identificados para la extracci\'on de palabras claves:
en base a aprendizaje supervisado y no supervisado, en el primero la computadora es
previamente entrenada con un conjunto de textos y palabras claves extraidas por
un experto ling\"uista para su posterior automatizaci\'on, mientras que en el segundo
la automatizaci\'on es totalmente independiente.\\

En este trabajo realizaremos la extracci\'on de palabras claves bajo un enfoque no
supervidado mediante el modelo \emph{TextRank} que utiliza algoritmos de clasificaci\'on
basados en grafos.\\

\section{Antecedentes}

% TODO: Describir herramientas existentes

\section{Planteamiento del problema}
Existe una cantidad considerable de material bibliogr\'afico que no cuenta con una
selecci\'on de palabras claves, lo cual dificulta su clasificaci\'on y posterior
consulta.

\section{Objetivos}

\subsection{Objetivo general}
Construir una herramienta software que permita obtener el conjunto de palabras claves
representativas de un texto mediante el Modelo TextRank.

\subsection{Objetivos espec\'ificos}
\begin{enumerate}
	\item Evaluar algoritmos de clasificaci\'on basados en grafos para ser aplicados en el
	Modelo TextRank.
	\item Evaluar los tipos de palabras que deben ser seleccionados para obtener mejores
	resultados.
\end{enumerate}

\section{Justificaciones}
\subsection{Justificaci\'on tecnol\'ogica}
\subsection{Justificaci\'on social}

% \section{Aportes}

% \section{Metodolog\'ia}

\section{Limites y alcances}
