\chapter{Marco Referencial}
\section{Introducci\'on}
A trav\'ez de la historia, el conocimiento humano ha sido almacenado en gran medida
de forma escrita, teniendose a los libros, peri\'odicos, art\'iculos cient\'ificos,
etc., como muestra de ello. En general este conocimiento ha sido escrito en lenguaje
natural. \\

Dada la abrumadora cantidad de este tipo de informaci\'on, es necesario su 
organizaci\'on para identificarla entre las diferentes \'areas de conocimiento 
a las que puede pertenecer, para este cometido es necesario contar con criterios 
de clasificaci\'on que faciliten su posterior consulta. \\

El Procesamiento del Lenguaje Natural (PLN), es una rama de la Ling\"uistica 
Computacional que b\'asicamente estudia la comunicaci\'on entre personas y 
m\'aquinas por medio de lenguajes naturales. Entre las principales tareas de las que 
se ocupa, se encuentra la \emph{extracci\'on de informaci\'on} que es utilizada 
para el reconocimiento de nombres de entidades y extracci\'on de terminolog\'ia. \\

La extracci\'on de terminolog\'ia permite identificar y seleccionar 
\emph{palabras claves} de un texto analizado, que representan su tema o 
motivo central, las cuales representan un criterio de clasificaci\'on eficiente.\\

Existen dos enfoques claramente identificados para la selecci\'on automatizada
de palabras claves: en base a aprendizaje supervisado y no supervisado, en el primero 
la computadora es previamente entrenada con un conjunto de textos y palabras 
claves seleccionadas por un experto ling\"uista para su posterior automatizaci\'on, 
mientras que en el segundo la automatizaci\'on es totalmente independiente.\\

En este trabajo realizaremos la selecci\'on de palabras claves bajo un enfoque no
supervidado mediante el modelo \emph{TextRank} que utiliza algoritmos de 
clasificaci\'on basados en grafos.\\

\section{Antecedentes}
El estado del arte actual en esta \'area esta representado principalmente por
m\'etodos de aprendizaje supervisado, Peter Turney fue el primero en sugerir uno,
el cual combina reglas heur\'isticas parametrizadas con algoritmos gen\'eticos
en un sistema conocido com \emph{GenEx} el cual identifica palabras claves en un
documento \cite{PT99}. \\

En 2003 Anette Hulth propone un m\'etodo de aprendizaje supervisado aplicando
conocimiento ling\"uistico al modelo (propiedades lexicas y sint\'acticas) para obtener
un mejor resultado \cite{AH03}. \\

En 2004 Rada Mihalcea y Paul Taran proponen un modelo no supervisado de extracci\'on
de palabras basado en algoritmos de clasificaci\'on basados en grafo \cite{RMPT04}. \\

Se ha identificado las siguientes herramientas software para la selecci\'on de
palabras claves:
\begin{description}[leftmargin=0cm]
	\item[Kea] Es un algoritmo de extracci\'on de frases implementado en Java,
	open-source, utilizado principalmente en indexaci\'on libre  o indexaci\'on con
	un vocabulario controlado \cite{KEA}.
	\item[Texlexan] Es un proyecto que permite el an\'alisis, clasificaci\'on y
	sumarizaci\'on de texto, teniendo la caracter\'istica principal que es 
	multiling\"ue \cite{TEXLEXAN}.
\end{description}

%TODO: Completar con otras herramientas

\section{Planteamiento del problema}
Existe una cantidad considerable de trabajos de Tesis en la carrera de Inform\'atica 
que no cuentan con una selecci\'on de palabras claves, lo cual dificulta su 
clasificaci\'on y consulta. \\

Se presenta la siguiente cuesti\'on: \\

?`La herramienta software para la selecci\'on automatizada de palabras claves
construida mediante el modelo TextRank, mejorar\'a el proceso de clasificaci\'on 
de los trabajos de Tesis de la carrera de Inform\'atica de la U.M.S.A.?

\section{Objetivos}

\subsection{Objetivo general}
Construir una herramienta software que permita obtener el conjunto de palabras claves
representativas de un texto mediante el Modelo TextRank.

\subsection{Objetivos espec\'ificos}
\begin{enumerate}
	\item Evaluar algoritmos de clasificaci\'on basados en grafos para ser su
	aplicaci\'on en el modelo TextRank.
	\item Evaluar los tipos de palabras que deben ser seleccionados para obtener mejores
	resultados.
\end{enumerate}

\section{Justificaciones}
\subsection{Justificaci\'on tecnol\'ogica}
El Procesamiento del Lenguaje Natural utiliza recursos computacionales especialmente
para el tratamiento probabil\'istico, debido a esto existen paquetes especializados
enfocados en este campo: \emph{Natural Language Toolkit} escrito en Python y
\emph{Apache OpenNLP} escrito en Java, que facilitan esta tarea.

\subsection{Justificaci\'on social}
Actualmente existe un cantidad considerable de material que no esta adecuadamente
clasificado, tales como ser las tesis y proyectos de grado en la carrera de 
Inform\'atica, que no presentan un conjunto de palabras claves, especialmente en
los m\'as antiguos que no tenian esa exigencia.
% \section{Aportes}

% \section{Metodolog\'ia}

\section{Limites y alcances}
Los trabajos de Tesis a ser tratados deben estar en formato digital, escritos en 
idioma espa\~nol.
