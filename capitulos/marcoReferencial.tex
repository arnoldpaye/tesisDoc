\chapter{Marco Referencial}
\section{Introducci\'on}
La comunicaci\'on entre humanos y m\'aquinas ha sido un tema de estudio muy amplio,
debido a su complejidad. \\

El Procesamiento del Lenguaje Natural (PNL), estudia m\'etodos para tratar 
el lenguaje natural propio de los seres humanos de forma autom\'atizada. \\

Entre las tareas m\'as importantes que trata el PNL tenemos: modelado del lenguaje,
extracci\'on de informaci\'on, generaci\'on de lenguaje, sumarizaci\'on de texto. \\

En este trabajo nos enfocaremos en la sumarizaci\'on de texto, para obtener resumenes
de texto de art\'iculos cient\'ificos, facilitando de esta manera la consulta de
material existente en internet. \\

El m\'etodo elegido para obtener los resumenes es el basado en el Modelo TextRank,
que utiliza algoritmos de calificaci\'on, modelando el texto como un grafo, donde
los v\'ertices son sentencias u oraciones del texto original, que despu\'es de
aplicarse el algoritmo tienen un puntaje, las sentencias con mayor puntaje
conformar\'an el resumen.

\section{Antecedentes}
Entre las herramientas existentes para realizar sumarios de texto, tenemos:
\begin{description}
	\item[MEAD] El proyecto MEAD es sumarizador multidocumento de texto que utiliza
	el algoritmo LexRank \cite{MEAD}.
	\item[Texlexan] Es una herramienta open source que realiza sumarizaci\'on,
	an\'alisis y clasificaci\'on de textos escritos en los siguientes idiomas: ingles,
	frances, aleman, italiano y espa\~nol \cite{TEXLEXAN}.
\end{description}
En la carrera de Inform\'atica existen los siguientes trabajos concernientes al tema:\\
\begin{description}
	\item[\cite{EZ01}] Elabora resumenes de texto mediante redes neuronales de texto en el
	\'ambito Sociol\'ogico.
	\item[\cite{HM05}] Realiza reconocimiento del lenguaje natural mediante agentes
	inteligentes.
\end{description}

% TODO: Describir herramientas existentes

\section{Planteamiento del problema}
Dada la abrumadora cantidad de informaci\'on existente en internet, se dificulta su
consulta al no contarse con resumenes que faciliten su compresi\'on.

\section{Objetivos}

\subsection{Objetivo general}
Construir una herramienta software para la elaboraci\'on de resumenes extractivos 
de art\'iculos cient\'ificos.  

\subsection{Objetivos espec\'ificos}
Los objetivos espec\'ificos se describen a continuaci\'on:
\begin{enumerate}
	\item Evaluar m\'etodos para extraer sentencias de texto de manera \'optima.
	\item Evaluar algoritmos de puntuaci\'on basados en grafos para aplicarlos al
	modelo TextRank.
\end{enumerate}

\section{Justificaciones}
\subsection{Justificaci\'on tecnol\'ogica}
\subsection{Justificaci\'on social}

% \section{Aportes}

% \section{Metodolog\'ia}

\section{Limites y alcances}
El presente trabajo se limitara a realizar sumarizaci\'on de texto escrito en lenguaje
espa\~nol monoling\"ue y en un formato extractivo.
