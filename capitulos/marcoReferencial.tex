\chapter{Marco Referencial}
\section{Introducci\'on}

\section{Antecedentes}
En la carrera de Inform\'atica existen los siguientes trabajos concernientes al tema:\\
\begin{description}
	\item[\cite{EZ01}] Elabora resumenes de texto mediante redes neuronales de texto en el
	\'ambito Sociol\'ogico.
	\item[\cite{HM05}] Realiza reconocimiento del lenguaje natural mediante agentes
	inteligentes.
\end{description}

\section{Planteamiento del problema}
Actualmente no se cuenta con una herramienta para realizar resumenes de texto escritos
en lenguaje natural en el \'ambito de los art\'iculos cient\'ificos.

\section{Objetivos}

\subsection{Objetivo general}
Elaborar una herramienta para realizar un sumario autom\'atico de texto extraidos de
articulos aplicando conceptos del \emph{Procesamiento del Lenguaje Natural} y
la \emph{Ling\"uistica Computacional} utilizando el algoritmo \emph{TextRank}.

\subsection{Objetivos espec\'ificos}
Los objetivos espec\'ificos se describen a continuaci\'on:
\begin{enumerate}
	\item Evaluar el algoritmo TextRank para realizar sumarizaciones de texto.
	\item Evaluar algoritmos para valorar las sentencias u oraciones de un texto para
	mejor el resultado del sumario.
\end{enumerate}

\section{Limites y alcances}
El presente trabajo se limitara a realizar sumarizaci\'on de texto escrito en lenguaje
espa\~nol monoling\"ue y en un formato extractivo.
