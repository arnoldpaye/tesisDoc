\chapter{Marco Te\'orico}

\section{El Sumario}
Un sumario es un texto que es producido a partir de uno o m\'as textos,
que contiene una porci\'on significante de la informaciones en el o los
textos originales, y que no excede la mitad del mismo. \cite{EH05} \\
	
\section{Sumarizaci\'on autom\'atica de texto}
Es el proceso de destilar la informaci\'on m\'as importante a partir de
una o m\'as fuentes para producir una versi\'on abreviada para un usuario
y tarea determinada realizado por una computadora. \cite{EL08} \\

Generalmente la fuente se centra en texto, pero tambi\'en puede ser
de otro tipo como ser: imagenes, video o audio asi como tambi\'en
informaci\'on online. Por otra parte se puede hablar de la sumarizaci\'on
de un solo documento o de multiples documentos, este ultimo se conoce
como Sumarizaci\'on Multi-documento (SMD) adem\'as los documentes fuentes
pueden estar en un solo lenguaje (monoling\"ue) o en diferentes lenguajes
(multiling\"ue). \\

El resultado de un sistema de sumarizaci\'on puede ser un \emph{extracto}
(selecci\'on de sentencias significativas) o una \emph{abstracci\'on}
(resumen del documento). \\

\subsection{Proceso de la sumarizaci\'on autom\'atica de texto}
Segun el enfoque de Sparck Jones \cite{KS99} el proceso se divide en las
siguientes etapas:
\begin{description}
	\item[$\bullet$ interpretaci\'on] del texto fuente para obtener una
	representaci\'on del mismo.
	\item[$\bullet$ transformaci\'on] de la representaci\'on del texto
	en una representaci\'on del sumario.
	\item[$\bullet$ generaci\'on] del sumario a partir de la representaci\'on
	del sumario.
\end{description}

\section{Lenguaje Natural}
Es el lenguaje hablado o escrito por humanos para prop\'ositos generales
de comunicaci\'on. Son aquellas lenguas que han sido generadas
espont\'aneamente en un grupo de hablantes con el prop\'osito de
comunicarse, a diferencia de otras lenguajes, como pueden ser un lenguaje
construido, lenguajes de programaci\'on o los lenguajes usados en el
estudio de la l\'ogica formal, especialmente la l\'ogica matem\'atica.

\section{Procesamiento del Lenguaje Natural}
El Procesamiento del Lenguaje Natural (PLN) es la disciplina encargada
de producir sistemas inform\'aticos que posibiliten la comunicaci\'on
hombre-computadora por medio del lenguaje humano a trav\'ez de la voz o
del texto escrito.

\section{M\'etodos de sumarizaci\'on}
Los m\'etodos de sumarizaci\'on se clasifican en dos categor\'ias:
\emph{m\'etodos extractivos} y \emph{m\'etodos abstractivos}
% TODO: Completar con otros tipos de m\etodos: basados en maximum-entropy,
% aided summarization

\subsection{M\'etodos extractivos de sumarizaci\'on}
Obtienen el sumario mediante la uni\'on de extractos del texto original,
que pueden ser palabras claves o sentencias completas.
% TODO: Citar m\'etodos

\subsection{M\'etodos abstractivos de sumarizaci\'on}
Obtienen el sumario mediante la generaci\'on de lenguaje natural, condensando
el texto en forma m\'as significativa que los m\'etodos extractivos.\\
% TODO: Citar m\'etodos

En este trabajo se va a utilizar un m\'etodo extractivo de sumarizaci\'on conocido
como \emph{Modelo TextRank}

\section{Modelo TextRank}
Es un m\'odelo de puntuaci\'on basado en grafos para el procesamiento de
texto \cite{RMPT04}, que es utilizado para la extracci\'on de palabras claves
y sentencias mediante algoritmos de puntuaci\'on basados en grafos.
% TODO: Completar este concepto

\subsection{Algoritmos de puntuaci\'on basados en grafos}
Son algoritmos que se utilizan para obtener valores significativos
(\emph{calificaci\'on}) de los nodos en un grafo. \\

Existen varios algoritmos de puntuaci\'on: \emph{PageRank} de Google,
\emph{HITS} de Kleinberg, \emph{Positional Function} de Herings \cite{RM04}. \\

A continuaci\'on se describe el algoritmo PageRank.

\subsubsection{Algoritmo PageRank}
% TODO: agregar gr\'afico
Dado $G=(V,E)$ un grafo dirigido, con un conjunto de v\'ertices $V$ y un conjunto
de arcos $E$, donde $E$ es un subconjunto de $V x V$, llamamos $In(V_i)$ al
conjunto de v\'ertices que apuntan a $V_i$ (predecesores) y $Out(V_i)$ al 
conjunto de v\'ertices que son apuntados por $(V_i)$ (sucesores). El puntaje del
v\'ertice $V_i$ esta dado por \cite{SBLP98}:

\begin{equation}
	S(V_i) = (1 - d) + d * \sum_{V_j\in In(V_i)}{\frac{1}{|Out(V_j)|}S(V_j)}
\end{equation}

donde $d$ es un factor de amortiguaci\'on que toma valores comprendidos entre
0 y 1, que expresa la probabilidad de avanzar de un v\'ertice a otro de manera
aleatoria, generalmente tiene el valor de 0.85 \cite{SBLP98}.

\subsection{Aplicaci\'on en la extracci\'on de sentencias} 
Para la aplicaci\'on en la extracci\'on de sentencias, el modelo TextRank realiza
las siguientes consideraciones antes de utilizar un algoritmo de de puntuaci\'on:
el grafo no es direccionado y los arcos son ponderados. \\

Aplicando las anteriores consideraciones en la ecuaci\'on (2.1) se tiene
\cite{RMPT04}:

\begin{equation}
	WS(V_i) = (1 - d) + d *
	\sum_{V_j\in In(V_i)}{\frac{w_{ij}}{\sum_{V_k\in Out(V_j)}{w_{jk}}}WS(V_j)}
\end{equation}

El grafo es construido de la siguiente manera: los v\'ertices son representados por
sentencias extraidas del texto bajo algun criterio ling\"uistico, existen arcos
entre las sentencias si existe una relaci\'on de similitud entre ellas.

En \cite{RMPT04} se presenta la siguiente relaci\'on de similitud: Dado dos
sentencias $S_i$ y $S_j$, $S_i$ esta compuesta por las palabras 
$w^i_1,w^i_2,...,w^i_{N_i}$ donde $N_i$ es la cantidad de palabras en $S_i$ y de igual
manera para $S_j$, la similitud entre $S_i$ y $S_j$ es:

\begin{equation}
	Similitud(S_i, S_j) = 
	\frac{|\{w_k|w_k\in S_i \wedge w_k\in S_j\}}{log(|S_i|)+log(|S_j|)}
\end{equation}

Es posible considerar otros criterios de similitud entre sentencias: n\'ucleo de
cadenas, similitud cosena, subsecuencia com\'un m\'as larga, etc. \\

El sumario se obtendra de la uni\'on de las sentencias con mayor puntuaci\'on
aplicando la ecuaci\'on (2.2) al grafo construido a partir de texto original.
